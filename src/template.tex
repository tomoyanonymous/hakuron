\documentclass[a4paper,openany]{jsbook}
\usepackage[includeheadfoot,top=20truemm,bottom=20truemm, right=25truemm,left=25truemm]{geometry} % 余白を調製

\usepackage{amsmath,amssymb}
\usepackage{bm}
\usepackage{ascmac}
\usepackage[dvipdfmx]{graphicx,color}
\usepackage{longtable} % 表組みに必要
\usepackage{booktabs} % 表組みに必要
\usepackage{subfig} % 図の横並びに必要
\usepackage{ulem} % 取り消し線に必要
\usepackage{url}

\usepackage{caption}

\usepackage{biblatex}
\usepackage{listings}
\usepackage{framed}
\usepackage[dvipdfmx]{hyperref}
\usepackage{pxjahyper} % ハイパーリンクに日本語が含まれる場合に必要
\hypersetup{% hyperrefのオプションリスト
    colorlinks=true,% カラーリンクを使用
    linkcolor=black,% 内部参照リンクの色
    citecolor=black,% 文献参照リンクの色
    filecolor=black,% ローカルファイル参照リンクの色
    urlcolor=blue % 外部参照URLの色
}

% 画像を記述された場所に挿入(デフォルトではページ上部に挿入されてしまう)
\usepackage{float}
\let\origfigure\figure
\let\endorigfigure\endfigure
\renewenvironment{figure}[1][2] {
    \expandafter\origfigure\expandafter[H]
} {
    \endorigfigure
}

% 画像の縦横比を保ったままサイズ変更を可能にする
\makeatletter
\def\maxwidth{\ifdim\Gin@nat@width>\linewidth\linewidth\else\Gin@nat@width\fi}
\def\maxheight{\ifdim\Gin@nat@height>\textheight\textheight\else\Gin@nat@height\fi}
\makeatother
\setkeys{Gin}{width=\maxwidth,height=\maxheight,keepaspectratio}

\def\tightlist{\itemsep1pt\parskip0pt\parsep0pt} % これがないと\tightlistが動かない

\newenvironment{Shaded}{\begin{shaded}}{\end{shaded}}
\definecolor{shadecolor}{RGB}{200,200,200}
\newenvironment{Highlighting}{}{}

\newcommand{\passthrough}[1]{#1}
\newcommand{\KeywordTok}[1]{\textcolor[rgb]{0.00,0.44,0.13}{\textbf{{#1}}}}
\newcommand{\DataTypeTok}[1]{\textcolor[rgb]{0.56,0.13,0.00}{{#1}}}
\newcommand{\DecValTok}[1]{\textcolor[rgb]{0.25,0.63,0.44}{{#1}}}
\newcommand{\BaseNTok}[1]{\textcolor[rgb]{0.25,0.63,0.44}{{#1}}}
\newcommand{\FloatTok}[1]{\textcolor[rgb]{0.25,0.63,0.44}{{#1}}}
\newcommand{\CharTok}[1]{\textcolor[rgb]{0.25,0.44,0.63}{{#1}}}
\newcommand{\StringTok}[1]{\textcolor[rgb]{0.25,0.44,0.63}{{#1}}}
\newcommand{\CommentTok}[1]{\textcolor[rgb]{0.38,0.63,0.69}{\textit{{#1}}}}
\newcommand{\OtherTok}[1]{\textcolor[rgb]{0.00,0.44,0.13}{{#1}}}
\newcommand{\AlertTok}[1]{\textcolor[rgb]{1.00,0.00,0.00}{\textbf{{#1}}}}
\newcommand{\FunctionTok}[1]{\textcolor[rgb]{0.02,0.16,0.49}{{#1}}}
\newcommand{\AttributeTok}[1]{\textcolor[rgb]{0.49,0.56,0.16}{#1}}
\newcommand{\RegionMarkerTok}[1]{{#1}}
\newcommand{\ErrorTok}[1]{\textcolor[rgb]{1.00,0.00,0.00}{\textbf{{#1}}}}
\newcommand{\NormalTok}[1]{{#1}}
\newcommand{\VariableTok}[1]{\textcolor[rgb]{0.10,0.09,0.49}{{#1}}}

\providecommand{\citep}{\cite}

\addbibresource{ref.bib}


\title{インフラストラクチャとしての音楽プログラミング言語mimiumの設計と実装}
\author{松浦知也}
\date{2022年3月1日}

\begin{document}
\maketitle
\tableofcontents

\hypertarget{ux5e8fux8ad6}{%
\chapter{序論}\label{ux5e8fux8ad6}}

\hypertarget{structure-of-the-thesis}{%
\section{structure of the thesis}\label{structure-of-the-thesis}}

\hypertarget{contribution-of-the-thesis}{%
\section{contribution of the thesis}\label{contribution-of-the-thesis}}

\hypertarget{ux97f3ux697dux30d7ux30edux30b0ux30e9ux30dfux30f3ux30b0ux8a00ux8a9eux7814ux7a76ux306eux7406ux8ad6ux7684ux306aux30d5ux30ecux30fcux30e0ux30efux30fcux30afux306eux63d0ux4f9b}{%
\subsection{音楽プログラミング言語研究の理論的なフレームワークの提供}\label{ux97f3ux697dux30d7ux30edux30b0ux30e9ux30dfux30f3ux30b0ux8a00ux8a9eux7814ux7a76ux306eux7406ux8ad6ux7684ux306aux30d5ux30ecux30fcux30e0ux30efux30fcux30afux306eux63d0ux4f9b}}

\hypertarget{ux30bdux30d5ux30c8ux30a6ux30a7ux30a2ux8a2dux8a08ux306bux304aux3051ux308bux30e1ux30c7ux30a3ux30a2ux7814ux7a76ux30a4ux30f3ux30d5ux30e9ux30b9ux30c8ux30e9ux30afux30c1ux30e3ux7814ux7a76ux306eux8996ux70b9ux306eux63d0ux4f9b}{%
\subsection{ソフトウェア設計におけるメディア研究、インフラストラクチャ研究の視点の提供}\label{ux30bdux30d5ux30c8ux30a6ux30a7ux30a2ux8a2dux8a08ux306bux304aux3051ux308bux30e1ux30c7ux30a3ux30a2ux7814ux7a76ux30a4ux30f3ux30d5ux30e9ux30b9ux30c8ux30e9ux30afux30c1ux30e3ux7814ux7a76ux306eux8996ux70b9ux306eux63d0ux4f9b}}

\hypertarget{limitations-not-mention}{%
\section{limitations, not mention}\label{limitations-not-mention}}

\hypertarget{live-coding}{%
\subsection{live coding}\label{live-coding}}

\hypertarget{latency-and-strict-realtimeness}{%
\subsection{latency and strict
realtimeness}\label{latency-and-strict-realtimeness}}

\hypertarget{background-a-computer-as-a-libre-machine}{%
\chapter{Background A: Computer as a Libre
machine}\label{background-a-computer-as-a-libre-machine}}

\hypertarget{keys-computer-as-a-meta-medium}{%
\section{Key's computer as a
meta-medium}\label{keys-computer-as-a-meta-medium}}

\hypertarget{programming-your-own-computer}{%
\subsection{``Programming your own
computer''}\label{programming-your-own-computer}}

\hypertarget{wisers-ubiquitous-computing}{%
\section{Wiser's Ubiquitous
computing}\label{wisers-ubiquitous-computing}}

\hypertarget{invisible-machines}{%
\subsection{invisible machines}\label{invisible-machines}}

\hypertarget{black-box-of-technologies}{%
\section{Black box of technologies}\label{black-box-of-technologies}}

\hypertarget{operating-system-and-time-sharing}{%
\section{Operating System and
Time-sharing}\label{operating-system-and-time-sharing}}

現代の、Dynabookよろしく可変的に作られたコンピューターはOSの上でユーザーアプリケーションが動作している。
この時、ユーザープログラムは自分の保存したメモリアドレスが実際の物理アドレスとしてどこにいるのかを知ることはできない。
また、ユーザープログラムはプリエンプティブマルチタスキングなOSの元では、プログラムがいつ実行されるのかを正確に知ることはできない。

キットラーのプロテクト・モード批判

\hypertarget{multi-tasking-and-general-purpose-computer}{%
\subsection{Multi-tasking and ``general-purpose''
computer}\label{multi-tasking-and-general-purpose-computer}}

\hypertarget{background-bmusical-medium}{%
\chapter{Background B:musical medium}\label{background-bmusical-medium}}

\hypertarget{ux30e1ux30c7ux30a3ux30a2ux30a4ux30f3ux30d5ux30e9ux30b9ux30c8ux30e9ux30afux30c1ux30e3ux306eux610fux56f3ux7684ux306aux8a2dux8a08ux306eux5fc5ux8981ux6027}{%
\section{メディアインフラストラクチャの意図的な設計の必要性}\label{ux30e1ux30c7ux30a3ux30a2ux30a4ux30f3ux30d5ux30e9ux30b9ux30c8ux30e9ux30afux30c1ux30e3ux306eux610fux56f3ux7684ux306aux8a2dux8a08ux306eux5fc5ux8981ux6027}}

\hypertarget{infrastructure-studies}{%
\subsection{infrastructure studies}\label{infrastructure-studies}}

\hypertarget{format-studies}{%
\subsection{format studies}\label{format-studies}}

sterne mp3 format matters \#\# today's situation of musical medium

\hypertarget{reproduction-tech-based-music-culture}{%
\subsection{reproduction-tech-based music
culture}\label{reproduction-tech-based-music-culture}}

\hypertarget{specifity-of-computer-music}{%
\section{specifity of computer
music}\label{specifity-of-computer-music}}

\hypertarget{object-based-format}{%
\section{object-based format}\label{object-based-format}}

\hypertarget{anti-musicking}{%
\section{anti-``musicking''}\label{anti-musicking}}

コンピューターが遍在していることを前提とした上で、新しい音楽フォーマットのあるべき形を検討する

\hypertarget{background-c-programming-language-for-sound-musicplfsam}{%
\chapter{Background C: Programming language for Sound
Music(PLfSaM)}\label{background-c-programming-language-for-sound-musicplfsam}}

\hypertarget{what-is-plfmas}{%
\section{What is PLfMaS?}\label{what-is-plfmas}}

\hypertarget{what-is-programming-language}{%
\subsection{What is Programming
Language?}\label{what-is-programming-language}}

\hypertarget{sampling-theorem}{%
\subsection{Sampling Theorem}\label{sampling-theorem}}

\hypertarget{stream-processing-and-timed-discrete-event}{%
\subsection{Stream processing and timed discrete
event}\label{stream-processing-and-timed-discrete-event}}

\hypertarget{computer-music-systemmusic-n}{%
\subsection{Computer music
system(MUSIC-N)}\label{computer-music-systemmusic-n}}

\hypertarget{elements-of-computer-music-language-by-dannenberg}{%
\section{6 Elements of Computer Music Language by
Dannenberg}\label{elements-of-computer-music-language-by-dannenberg}}

\begin{itemize}
\tightlist
\item
  Syntax
\item
  Semantics
\item
  Runtime

  \begin{itemize}
  \tightlist
  \item
    implicitly means ``compiler''(trade-off according to IR)
  \end{itemize}
\item
  Library
\item
  IDE
\item
  Community
\end{itemize}

\hypertarget{how-can-we-evaluate-the-programming-language-for-sound-as-good}{%
\section{How can we evaluate the programming language for sound as
``good?''}\label{how-can-we-evaluate-the-programming-language-for-sound-as-good}}

\hypertarget{terminology}{%
\subsection{terminology}\label{terminology}}

\hypertarget{ambigious-use-of-general-efficient-and-expressive}{%
\subsection{ambigious use of ``general'', ``efficient'' and
``expressive''}\label{ambigious-use-of-general-efficient-and-expressive}}

\begin{itemize}
\tightlist
\item
  Lazzarini2013
\item
  Brandt2002(Chronic's Tradeoff)
\end{itemize}

\hypertarget{from-the-terminology-of-general-purpose-languagecoblenz}{%
\subsection{from the terminology of general-purpose
language(Coblenz)}\label{from-the-terminology-of-general-purpose-languagecoblenz}}

\hypertarget{model-of-human-in-the-loop-system-in-music-programming.}{%
\subsection{model of human-in-the-loop system in music
programming.}\label{model-of-human-in-the-loop-system-in-music-programming.}}

\hypertarget{situating-cultural-background-to-the-language-spec}{%
\subsection{situating cultural background to the language
spec}\label{situating-cultural-background-to-the-language-spec}}

\begin{itemize}
\tightlist
\item
  CSound's layering ``Score'' ``Orchestra'' ``Instrument''
\item
  How can we generalize the language as possible?
\end{itemize}

\begin{quote}
The commercial musical software we know today (all made in the West)
perform abstractions and generalisations from the perspective of certain
popular styles of Western music. (Magunusson thesis 2009)
\end{quote}

\hypertarget{designing-mimium}{%
\chapter{Designing mimium}\label{designing-mimium}}

\hypertarget{ux57faux672cux6587ux6cd5}{%
\section{基本文法}\label{ux57faux672cux6587ux6cd5}}

\begin{Shaded}
\begin{Highlighting}[]
\KeywordTok{fn}\NormalTok{ hoge\{}
\NormalTok{    fuga = 2;}
\NormalTok{    return 0}
\NormalTok{\}}
\end{Highlighting}
\end{Shaded}

\hypertarget{type-systems}{%
\section{Type Systems}\label{type-systems}}

\hypertarget{scheduling-with}{%
\section{scheduling with @}\label{scheduling-with}}

\hypertarget{signal-procesing-with-self}{%
\section{signal procesing with self}\label{signal-procesing-with-self}}

\hypertarget{example-codes}{%
\section{Example Codes}\label{example-codes}}

\hypertarget{implementing-mimium}{%
\chapter{Implementing mimium}\label{implementing-mimium}}

\hypertarget{overall-architecture}{%
\section{overall architecture}\label{overall-architecture}}

\begin{figure}
\hypertarget{fig:arch}{%
\centering
\includegraphics{img/mimium-arch-v4-affinity.pdf}
\caption{mimium実行環境アーキテクチャ。}\label{fig:arch}
}
\end{figure}

mimiumの実行環境の基本的アーキテクチャを図~\ref{fig:arch}に示す。

\hypertarget{implementation-of-the-compiler}{%
\section{implementation of the
compiler}\label{implementation-of-the-compiler}}

\hypertarget{parser}{%
\subsection{parser}\label{parser}}

\hypertarget{internal-representation-1lambda-like-ir}{%
\subsection{Internal representation 1(lambda-like
IR)}\label{internal-representation-1lambda-like-ir}}

\hypertarget{internal-representation-2ssa-imperative-ir}{%
\subsection{internal representation 2(SSA-imperative
IR)}\label{internal-representation-2ssa-imperative-ir}}

\hypertarget{state-variable-analysis}{%
\subsection{State Variable Analysis}\label{state-variable-analysis}}

\hypertarget{implementation-of-the-runtime}{%
\section{implementation of the
runtime}\label{implementation-of-the-runtime}}

\hypertarget{scheduler}{%
\subsection{scheduler}\label{scheduler}}

\hypertarget{jit-engine}{%
\subsection{JIT engine}\label{jit-engine}}

\hypertarget{audio-driver}{%
\subsection{audio driver}\label{audio-driver}}

\hypertarget{implementation-of-frontend}{%
\section{implementation of frontend?}\label{implementation-of-frontend}}

\hypertarget{reflections-from-the-design-and-implementation}{%
\chapter{Reflections from the design and
implementation}\label{reflections-from-the-design-and-implementation}}

\hypertarget{limitations}{%
\section{Limitations}\label{limitations}}

\hypertarget{combination-of-scheduling-and-signal-processing}{%
\subsection{Combination of Scheduling and Signal
Processing}\label{combination-of-scheduling-and-signal-processing}}

\hypertarget{parametric-replication-of-signal-processor}{%
\subsection{Parametric Replication of signal
processor}\label{parametric-replication-of-signal-processor}}

\hypertarget{implications}{%
\section{implications}\label{implications}}

\hypertarget{multi-stage-computation}{%
\subsection{multi-stage computation}\label{multi-stage-computation}}

\hypertarget{appendix-a---history-of-sound-programming-language}{%
\chapter{Appendix A - History of Sound Programming
Language}\label{appendix-a---history-of-sound-programming-language}}

\hypertarget{sampling-theorem-1}{%
\section{Sampling Theorem}\label{sampling-theorem-1}}

引用テスト\citep{Magnusson2009}


test

\printbibliography[title = 参考文献]

\end{document}